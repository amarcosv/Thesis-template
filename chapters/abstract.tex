\chapter*{Abstract}
\addcontentsline{toc}{chapter}{Abstract}
\markboth{Abstract}{Abstract} 
Light microscopy in the \acrfull{nir2} window has shown to be a very promising approach to increase the depth of penetration and reduce the loss of resolution of deep tissue images in \acrfull{lsfm} and \acrfull{opt}.
In this spectral window the scattering of light in biological tissue samples is much weaker than in the visible, where this phenomenon limits the penetration to a few hundreds of microns for \textit{in vivo} specimens and non cleared tissues.

However, measurements in the \gls{nir2} demonstrated to fall in the yet uncharacterized ballistic to diffusion transition, where neither the \acrfull{da} nor the ballistic model can accurately predict the propagation of light. This thesis investigates the performance of \gls{opt} and \gls{lsfm} in this spectral window and equivalent low scattering media through the development of a set of computational tools and experimental approaches to characterize the penetration and challenges of imaging in this regime. 

The characterization begins with an evaluation of the validity of the \gls{da} in low scattering media using a new \acrfull{mc} simulation method to estimate the forward flux density of point and collimated sources in infinite media. The proposed tool allows to demonstrate the weaknesses of the \gls{da} to model light propagation below a transport mean free path.

In order to study \gls{lsfm} in the \gls{nir2} this work presents a \gls{mc} simulator capable of mimicking the entire photon flow of a light sheet microscope. The code was developed with the combination of a modified version the validated Monte Carlo eXtreme (MCX) package and a novel algorithm that focuses the detected fluorescence and computes the optical sections according to the position of the light sheet in the volume. The tool simulates light sheet acquisitions of a distribution of fluorophores in a large volume with the optical properties of tissues from several spectral windows. The results show that \gls{lsfm} is expected to resolve structures at depths of at least one transport mean free path in scattering media, which demonstrates that the predictions of the theoretical framework can be translated into this optical imaging modality. 

The \gls{lsfm} simulator showed to be also useful during the validation step of new optical imaging modalities. Minor modifications to the package enable its use as a tool to assess the performance of Statistical Projection Optical Tomography (SPOT), a new tomographic imaging technique in which projections from different views are acquired through the integration of the stack of z planes of a \gls{lsfm} volume. The simulation software was used to compare the results from this method against traditional \gls{lsfm} images, demonstrating isotropic voxel size and a more stable resolution at deep z planes for SPOT. Moreover, the inhomogeneities of the illumination due to the attenuation of the \gls{ls} are suppressed with this new technique. 

The last section introduces an experimental proof of concept version of a transmission \gls{opt} system in the \gls{nir2}. Since the increase in the absorption coefficient of water in this window prevents the use of conventional illumination schemes, this work explores two alternative approaches to overcome this limitation. The first is based on polarized light and the second implements a scanning method to illuminate the sample selectively. In both cases reconstructions with \gls{nir2} light retrieved contrast from absorption from thick non cleared samples.
